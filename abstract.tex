This final year project is an evaluation of the effect that frameworks have on the development of a web application. The frameworks use primarily within this project were Spring Model View Controller, Hibernate ORM and Apache Tiles. The web application developed for this project is a web presence for Monaleen Tennis Club. This application will be evaluated with cognisance of non functional requirements, specifically security, productivity, performance, extensibility and usability.

The methodology used within this projects focuses on requirements elicitation, system design, implementation, testing, deployment, along with the use of metrics and visualisation to support software quality. Tools such as Maven are used to organise the project, and manage the 30 dependencies required for this application. A source control solution, GitHub, is used to manage different versions of the code base of over 50 classes and 7000 lines of code. The support for metrics is provided by an Spring Tool Suite plugin called \textit{Metrics}, and is further analysing using a tool by intoolitus, inFusion, to identify potential issues in the design within the code. CodeCity is used to provide a visualisation of the code base to highlight the structure of the application, and highlight important classes within it. Gource, an open source tool, is used to visualise the work flow of the application over the course of its development cycle. 

The non-functional requirements identified within this project were evaluated as part of this project. The OWASP Top 10 were used to evaluate security, with Spring providing support for all 10, either by default or with little developer interaction required. Productivity was increased through the use of Hibernate by reducing both lines of code needed for data access and eliminating the need for potential complex SQL queries. Hibernate also performed better than JDBC, the standard method for Java database connectivity, in all tests performed. The use of Hibernate again over JDBC supported extensibility by enforcing the separation of the service and DAO layers through the use of entity class annotations. The use of Apache Tiles in conjunction with Spring MVC supported usability by allowing the developer to produce a uniform design across the web application. These evaluations show that there were benefits found to using a framework with all of the specified non-functional requirements. 

The application, backed by this report, shows that there are a number of benefits to incorporating web application frameworks into the development process, which impact





