\chapter{Conclusions}
\label{conclusion}

\section{Summary}

\section{Key Findings}

\section{Future Work}

\subsection{Hibernate in Design}
Hibernate was a very powerful framework throughout the development of this project. It was very much a case of learning by doing. Having accumulated experience with the framework, I would like to apply it more in the design stage on an application. One idea that came to me late in the project was the idea of having a top level class that could build the entire application, as depicted in Figure~\ref{fig:toplevel}.


\begin{lstlisting}
@Entity
public class WebApplication(){

@ElementCollection
@CollectionTable (name = "tournament", joinColumns=@JoinColumn(name="id"))
private Tournaments tournaments;

@ElementCollection
@CollectionTable (name = "timetable", joinColumns=@JoinColumn(name="id"))
private Timetable timetable;

@ElementCollection
@CollectionTable (name = "news", joinColumns=@JoinColumn(name="id"))
private News news;

@ElementCollection
@CollectionTable (name = "events", joinColumns=@JoinColumn(name="id"))
private Events events;

@ElementCollection
@CollectionTable (name = "users", joinColumns=@JoinColumn(name="id"))
private User users;

public WebApplication(){
	create();
}

private void create(){
	tournaments = getTournaments();
	etc.....
}

//getters and setters

\end{lstlisting}
\begin{figure}[H]
\caption{'Top Level' Class}
\label{fig:toplevel}
\end{figure}

(This section needs to be rewritten. I'm tired right now so just doing stream of conciousness) 

It would be possible to build the application so that all classes were linked through a hierarchy to one root class. This may impact performance, but it would allow a number of benefits such as ease of deployment, and the ability to take snapshots of the application at any stage.

I also would have like to use Hibernate with more complicated objects rather than just collection classes, but time constraints did not allow this.

\subsection{Tournament Business Logic}

The tournament section of the site is something that I would like to have done a bit more on, but unfortunately, due to the time constraints within a FYP, a decison had to be made to focus more on either the timetable or the tournament. The timetable provides a service for all members of a club, not just those playing competitively. In the interim, third party solutions such as \textit{http://ti.tournamentsoftware.com} exist. The tournament sorting mechanism was designed so that the logic of sorting members into teams is encapsulated in one method in an interface. Implementing a new version would not require any significant changes to the application. 

\subsection{Web Services and HTML5}
Other aspects of a web application, such as web services and HTML5 were omitted from the final submission due to time constraints, but are definitely something that I would have an interest in looking at later on. Web services were touched on briefly, through the exposure of part of information about the timetable. 

\subsection{Timetable}
While the timetable works well in its current state, I would like to optimize the code to be more readable. I experimented with the use of key-map objects later in the development cycle, and beleive the use of these could reduce the code size in the View by at least 60\%, and help compartmentalize two or three methods into one.

\section{Personal Perspective}