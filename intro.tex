\chapter{Introduction}
\label{intro}

\section{General Introduction}

This project concerns the development of a web application using a web framework in conjunction with a number of other tools. Throughout development, there is a particular cognisance towards the support of Non-Functional Requirements [NFRs] by both the web framework and the supporting tools throughout the development process.


\subsection{General Introduction}

The main goal of this project is to reflectively analyse a WAF [Web Application Framework], and architecture stack, in the creation of a website. This will be analysed in respect to both functional and non-functional requirements. Two key requirements are extensibility and maintenance. Extensibility refers to the ability of the framework to allow added functionality to the web application without having to modify the core workings of the application. Maintenance refers to the upkeep of the code, and facilitates the modification of the source code after the product is deployed. This may be to correct faults, improve attributes such as performance and security.
The creative driver of the project is the development of a website to meet the requirements and needs of Monaleen Tennis Club, for both members of the club and of the committee. These needs will overlap as all committee representatives are all club members, but not all members are on the committee. From this, it was important to identify the precise requirements for each type of user. The main focus of this project was for the club to be able to perform their core functions through the website. This extended to the registration of members, a timetable for the courts, the creation and distribution of tournament schedules, the organisation and timetabling of training sessions, a method to contact all members and a news section to update and advise members of changes and upcoming events .  

\begin {itemize}
\item	Member Management
\item	Timetable Management
\item	Tournament Management
\end{itemize}


\section{Objectives}

\section{Scope}


\section{Methodoloy}

The methodology chosen as the foundation for this project is the Russo and Graham (1998) design methodology. It focuses on 9 iterative steps, each with feedback loops. The steps are outlined below


\begin {itemize}
\item Identification of the problem
\item Analysis
\item Design of the Application
\item Resource Gathering
\item Coding
\item Testing
\item Implementation
\item Post Implementation Review and Maintainance
\end{itemize}

Other methodologies that were examined such as Balasubramanin and Bashian (1997), Siegel (1997), Iskawitz et al (1995) and Cranford-Teague (1998). The pros and cons of these methodologies were examined by Howcroft and Carroll (Howcroft and Carroll 2000), and after an examination of their findings, the Russo and Graham methodology best suited the nature and scale of this project. While the other methodologies are strong, they are geared towards large scale web development projects, or towards document-centred websites, and would not suit this project.(Howcroft and Carroll 2000)
Using these as a guide, the following methodology was established.

\begin {itemize}
\item Identification of the problem
\item Structured Literature Review
\item Statement of the FYP Objectives
\item Design of the Test Suite
\item Development of the Prototype
\subitem Analysis
\subitem Design of the Application
\subitem Resource Gathering
\subitem Design Review
\subitem Coding
\subitem Testing
\subitem Implementation
\subitem Post Implemetation Review and Maintainance
\item Emperical Study
\item Critical Evaluation of the Results
\end{itemize}


\section{Overview of Report}



\section{Motivation}

The motivation behind this project for me was to examine, understand and work with software frameworks and methodologies that would be commonly used in industry, and to develop a software application from them. The module, Distributed Systems, touched on some of the tools and technologies, Netbeans and EJB respectively, used in relation to Java Enterprise development, and this formed the foundation of my interest in the area. I felt the FYP was a perfect vehicle to supplement my knowledge of this subject, with particular attention being paid to popular and in demand technologies. 

