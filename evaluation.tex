\chapter{Evaluation}
\label{evaluation}

\section{Security}

\subsection{OWASP Vulnerabilities}

In analysing support for security, the Open Web Application Security Project [OWASP] was examined. OWASP provides a list of the top 10 vulnerabilities that exist within web applications. The following vulnerabilities are addressed directly by the frameworks used in this project.

\begin{enumerate}
\item Injection
\begin{itemize}
\item Injection occurs when "untrusted data is sent to an interpreter as part of a command or query" \parencite{owasp2013}. This can allow unauthorised access to a database, and the exposure of the data within. Hibernate aids the prevention of Injection by providing support for named parameters. Hibernate further prevents injection by eliminating SQL queries entirely, and by creating objects through mapping of attributes to columns within the database. JDBC does not support the use of named parameters natively, but Spring does provided a \textit{NamedParameterJdbcTemplate} class which provides this functionality to the JDBC class. 

\end{itemize}

\item Broken Authentication and Session Management
\begin{itemize}
\item This occurs when "application functions related to authentication and session management are often not implemented correctly" \parencite{owasp2013}. When developers create their own session management schemes, they will inadvertently introduce flaws. Spring provides robust session management and authentication techniques through the use of EBAC, as discussed earlier. This allows the developer to focus more on the design of the application, and the business logic within, without having to be concerned with created a strong security solution.
\end{itemize}

\item Cross Site Scripting (XSS)
\begin{itemize}
\item XSS is "whenever an application takes untrusted data and sends it to a web browser without proper validation or escaping" \parencite{owasp2013}. This allows the potential use of malicious scripts in a victims browser, which may enable an attacker to gain access to the user session. Spring provides an extension of the Java EE 6 class \textit{HttpServletRequestWrapper} called \textit{XSSWrapper}. This class takes a HTTP Request object, and checks the values of getParamters(), getParameterValues() and getHeaders(). The XSSWrapper object then performs a stripXSS() method that checked the validity of the parameters with an HttpServletRequest object, and in the event that an attempted XSS attack is present, nullifies it by escaping special characters.
\end{itemize}

\item Security Misconfiguration
\begin{itemize}
\item Security Misconfiguration is not "having a secure configuration defined and deployed for the application framework" \parencite{owasp2013}. Within Spring MVC, this refers to the configuration for EBAC. An example of poor configuration would be to implement this system by denying access to URL mappings, rather than denying access to all and allowing exceptions. While it is possible to deny access to relevant mappings, it is more likely that a mistake would slip through. A breakdown of the mappings organised by the type of access is depicted in Table~\ref{fig:accesstable}. permitAll refers to mappings that allow access to all users, isAuthenticated requires that the user be logged in, and admin requires that a user be logged in and possess an administrator role.

\begin{table}[H]
\caption{EBAC Breakdown}
\begin{center}
    \begin{tabular}{ | l | l | l | p{5cm} |}
    \hline
	\textbf{permitAll} & \textbf{isAuthenticated}& \textbf{admin}\\ \hline
	23 & 13 & 50\\ \hline
    \end{tabular}
\end{center}
\label{fig:accesstable}
\end{table} 
\end{itemize}

\item Sensitive Data Exposure
\begin{itemize}
\item Sensitive Data Exposure deals with the encryption of sensitive user information such as passwords, credit card details and anything that made lead to identity theft. This data "deserves extra protection such as 
encryption at rest or in transit" \parencite{owasp2013}. Within the scope of this application, the main concern was user passwords, as no payment information is held. Spring provides methods for encoding data using 256bit AES encryption. The encrypted value is stored in the database, not the actual password. When the user logs in, the plaintext password is encrypted  using a key unique to that user. The encrypted password is then compared to the stored value. Each user has a unique key to generate the password so that even if two users have the same password, it will not appear the same in the stored values. 
\end{itemize}

\item Unvalidated Redirected and Forwards
\begin{itemize}
\item Within any web application, they will be scope for redirects. For example, if a user logs in, it is normal to redirect them to the home page. However, the application must be cognisant of where the redirect is coming from, and ensure its validity. "Without proper validation, attackers can redirect victims to phishing or malware sites, or use forwards to access unauthorized pages" \parencite{owasp2013}. The ViewResolver, provided in Spring and extended by Apache Tiles, is responsible for validating this information. All redirected are passed to the ViewResolver as a string. If a value corresponding to this string is not found within the ViewResolver configuration, an error will be returned in the form of a 404 Page Not Found value or an access denied value, as defined in the security-context file. 
\end{itemize}
\end{enumerate}

\subsection{Security Bugs}

A significant bug was discovered within the application during user testing. When a user was editing their profile, a model attribute was used to map the data changed in the View to create an object within the Controller. The application did not allow the user to edit all attributes within the user class, such as their role within the application. As a result, any changes to a user object where the attributes were not defined within the view resulted in attributes not present in the view being set as null. This meant that a user would have no role within the system, as the role attribute would be changed to null. To overcome this, a hidden value mapped to the role attribute was inserted into the form. This fixed the problem, but in doing so, introduced a significant security risk. 

Using a tool to intercept HTTP Requests, such as WebScarab, it was possible to change the hidden field. This would allow a regular user to change their role from ROLE MEMBER to ROLE ADMIN, giving them full access to the site. This happened as the object was not checked before being persisted by the controller, service and dao layers. This method was tested using the WebScarab tool, and the vulnerability was confirmed. WebScarab is a framework that analysing applications that communicate using HTTP and HTTPS protocols, and was used to intercept GET and POST requests in this application. In order to circumvent this, the role of the currently logged in user was retrieved from the framework, and this value was saved, rather than a value defined within the view. This example showed that while the framework does provide a considerable layer of security, misuse and bad programming can expose the application. 

\section{Extensibility}

The goal of extensibility is "controlling the time and costs to implement, tests and deploy changes" \parencite{bass2003software}. The use of Hibernate in the architectural stack provides a measure of extensibility support for changes in entity classes. Any addition to an entity class would result in a change to a database schema. 

An application using Hibernate, in conjunction with Spring MVC, would need only to modify the View, illustrated in Figure~\ref{fig:viewChange}. This would need to be changed to facilitate input of the new attribute. Spring MVC uses spring form tags to map values to the object within the Controller, and the DAO layer persists the object. Since the object is configured with database mapping at class level, no changes are necessary to the DAO. The addition of Line 14 is Figure~\ref{fig:viewChange} is the only changed needed to allow for the attribute to be modified within the application. The Controller, Service layer and DAO would remain unchanged.

\begin{table}[H]
\begin{lstlisting}
<sf:form id="details" method="post" action="${pageContext.request.contextPath}/adminProfileUpdate" commandName="member">
<table class="formtable">
Name:<sf:input name = "name" path="name" type="text"/>
Email:<sf:input name = "username" path="username" type="text"/>
Password:<sf:input name = "password" path="password" type="text"/>
Address Line 1<sf:input name = "ad_line1" path="ad_line1" type="text"/>
Address Line 2<sf:input name = "ad_line2" path="ad_line2" type="text"/>
City/Town<sf:input name = "ad_city" path="ad_city" type="text"/><br/>
County<sf:input name = "ad_county" path="ad_county" type="text"/><br/>
Contact Number<sf:input name = "contact_num" path="contact_num" type="text"/>
Emergency Contact Name<sf:input name = "em_con_name" path="em_con_name" type="text"/>
Emergency Contact Number<sf:input name = "em_con_num" path="em_con_num" type="text"/>
Role<sf:select path="authority"><sf:options items="${roles }"/></sf:select>
New Attribute<sf:input name = "new_attribute" path="new_attribute" type="text"/> //new line added
<input type="hidden" value="${member.username }" name="username"/>
<input value="Change Details" type="submit"/>
\end{lstlisting}
\caption{Change to view to support new attributes}
\label{fig:viewChange}
\end{table}
Using JDBC, again in partnership with Spring MVC, the View and the DAO object would need to be changed. Within the DAO, two actions must be performed, as depicted in Figure~\ref{fig:jdbcChanges}. Line 18 shows the addition of an extra parameter to map the value to a position in the SQL statement. The SQL statement must be modified also in order to correctly insert or update the persisted value in the database. This is shown in Lines 23 through 26, with the changes being shown in capital letters. 

\begin{lstlisting}
@Transactional
public boolean createUserJDBC(User user) {
MapSqlParameterSource params = new MapSqlParameterSource();
params.addValue("username", user.getUsername());
params.addValue("name", user.getName());
params.addValue("password", passwordEncoder.encode(user.getPassword()));
params.addValue("gender", user.getGender());
params.addValue("member_type", user.getMember_type());
params.addValue("grade", user.getGrade());
params.addValue("ad_line1", user.getAd_line1());
params.addValue("ad_line2", user.getAd_line2());
params.addValue("ad_city", user.getAd_city());
params.addValue("ad_county", user.getAd_county());
params.addValue("contact_num", user.getContact_num());
params.addValue("em_con_name", user.getEm_con_name());
params.addValue("em_con_num", user.getEm_con_num());
params.addValue("role", "ROLE_MEMBER");
params.addValue("NEW_ATTRIBUTE", "user.getNEW_ATTRIBUTE());
return jdbc
	.update("insert into users (username, name, password, gender, 
	member_type, grade, ad_line1, ad_line2, ad_city, ad_county,
	contact_num, em_con_name, em_con_num, enabled, authority,
	bookings_left, NEW_ATTRIBUTE) values (:username, :name, :password, 
	:gender, :member_type, :grade, :ad_line1, :ad_line1, 
	:ad_city, :ad_county, :contact_num,:em_con_name, :em_con_num,
	false, :role, 0, :NEW_ATTRIBUTE)",params) == 1;
	}
\end{lstlisting}
\begin{table}[H]
\caption{JDBC DAO changes with entity change}
\label{fig:jdbcChanges}
\end{table}

Hibernate provides better support for extensibility in this regard, allowing the developer to focus on database and view changes, with no need to modify the persistence layer between the application and the database.

\section{Usability}

The deliverable from this project was subjected to the same usability criteria as illustrated in Section~\ref{sec:usabilty}. The results are shown in Tables~\ref{fig:projecttable1} through~\ref{fig:projecttable5}. The overall results are compared in Figure~\ref{fig:projecttable6}.

\begin{table}[H]
\caption{Table 1 - Links}
\begin{center}
    \begin{tabular}{ | l | l | p{5cm} |}
    \hline
\textbf{Criteria} & \textbf{Project Application}\\ \hline
	Links are updated? & Yes\\ \hline
	Short-cuts for frequent users? & Yes\\ \hline
	Warning if link leads to large file? & n/a\\ \hline
	Indication of restricted access for link & Yes \\ \hline
	Link text indicates nature of target & Yes \\ \hline
	Total & 4/5 \\ \hline	
    \end{tabular}
\end{center}
\label{fig:projecttable1}
\end{table}

\begin{table}[H]
\caption{Table 2 - Feedback Mechanisms}
\begin{center}
    \begin{tabular}{ | l | l | p{5cm} |}
    \hline
\textbf{Criteria} & \textbf{Project Application}\\ \hline
	Contact Details for website maintainer & Yes \\ \hline
	Link to page maintainer on each page & Yes \\ \hline
	Forms provided for users to enter details & Yes\\ \hline
	FAQ provided & Yes \\ \hline
	Feedback mechanisms are fully operational & Yes\\ \hline
	Total & 5/5 \\ \hline	
    \end{tabular}
\end{center}
\label{fig:projecttable2}
\end{table}

\begin{table}[H]
\caption{Table 3 - Accessability}
\begin{center}
    \begin{tabular}{ | l | l | p{5cm} |}
    \hline
\textbf{Criteria} & \textbf{Project Application}\\ \hline
	Speed of site adequate& Yes\\ \hline
	Site Availability High? & Yes\\ \hline
	Website made known through search tools** & n/a \\ \hline
	Link back to parent entity & Yes\\ \hline	
	Name of entity reflected in URL & Yes \\ \hline	
	URL is not over complex,likely to be mistyped & Yes \\
	\hline	
	Total & 5/8\\ \hline	
    \end{tabular}
\end{center}
\label{fig:projecttable3}
\end{table}


\begin{table}[H]
\caption{Table 4 - Design}
\begin{center}
    \begin{tabular}{ | l | l | p{5cm} |}
    \hline
\textbf{Criteria} & \textbf{Project Application}\\ \hline
	Format,Graphic Design Appropriate & Yes \\ \hline
	Pages appropriate length, clear, uncluttered& Yes \\ \hline
	Consistent format through website & Yes \\ \hline
	Site compatible with multiple browsers & Yes\\ \hline
	Site can be used without graphics & No\\ \hline
	Total & 4/5 \\ \hline	
    \end{tabular}
\end{center}
\label{fig:projecttable4}
\end{table}

\begin{table}[H]
\caption{Table 5 - Navigability}
\begin{center}
    \begin{tabular}{ | l | l | p{5cm} |}
    \hline
	\textbf{Criteria} & \textbf{Project Application}\\ \hline
	Website organised by anticipated user need & Yes\\ \hline
	Navigation options are distinct& Yes \\ \hline
	Conventional navigation models*& Yes\\ \hline
	Navigation links are provided on all pages & Yes\\ \hline
	Browsing facilitated by menu or site map &Yes \\ \hline
	Reach any point in appropriate number of clicks** & Yes\\ \hline
	Search Engine Provided & No\\ \hline
	Total & 6/7 \\ \hline	
    \end{tabular}
\end{center}
\label{fig:projecttable5}
\end{table}
\textit{*eg menu on top or left of screen}\newline
\textit{**defined as 3,} \cite{smith2001applying}

\begin{table}[H]
\caption{Table 6 - Overall Results}
\begin{center}
    \begin{tabular}{ | l | l | p{5cm} |}
    \hline
	Site & Total\\ \hline
	Monaleen Tennis Club & 14/30\\ \hline
	Tralee Tennis Club & 22.5/30\\ \hline
	Project Application & 24/30\\ \hline	
    \end{tabular}
\end{center}
\label{fig:projecttable6}
\end{table}

\section{Productivity and Performance}

\subsection{Hibernate}

One of the core technologies used within the application was Hibernate. One of the main benefits Hibernate provided to the project was simplifying the code in order to persist objects. When developing the application, the only concern was with objects as a whole, not on their individual attributes. With JDBC, there was a need to use SQL syntax within the application. In the case of the User class, there were 16 attributes which made the SQL INSERT statement for that class, shown in Figure~\ref{fig:jdbcsql}, difficult to read and troubleshoot.

\begin{figure}[H]
\begin{lstlisting}
@Transactional
public boolean createUserJDBC(User user) {
	MapSqlParameterSource params = new MapSqlParameterSource();
		//other params cut out to save space
		params.addValue("role", "ROLE_MEMBER");
		return jdbc
			.update("insert into users (username, name, password, gender, 
			member_type, grade, ad_line1, ad_line2, ad_city, ad_county,
			contact_num, em_con_name, em_con_num, enabled, authority,
			bookings_left) values (:username, :name, :password, 
			:gender, :member_type, :grade, :ad_line1, :ad_line1, 
			:ad_city, :ad_county, :contact_num,:em_con_name, :em_con_num,
			false, :role, 0)",params) == 1;
	}
\end{lstlisting}
\caption{JDBC SQL INSERT}
\label{fig:jdbcsql}
\end{figure}

However, it could be arguing by supplying a framework, such as Hibernate, to automate the creation of a query language that performance of an application could be affected. To examine this briefly, a small test was created. This this test, the application would use JDBC to create 10,000 records, update those records and then delete them. The application would then perform the same actions with the same data using Hibernate. The time taken to perform these tasks as a whole would be recorded in nanoseconds. The test is illustrated in Figure~\ref{fig:jdbcvhibernate} below.

\begin{figure}[H]
\begin{lstlisting}
@RequestMapping("/testdatabases")
public String testdatabases(Model model){
	User user = new User(); // create a new user
	long jStart = System.nanoTime();
	model.addAttribute("jdbcStart", jStart);
	for (int i = 0; i < 10000; i++){
		user.setUsername(i + "@test.ie");
		userService.createJDBC(user);
		userService.enableUserJDBC(user);
		userService.deleteUserJDBC(user);
	}
	model.addAttribute("jdbcFinish", (System.nanoTime() - jStart));
	
	long hStart = System.nanoTime();
	model.addAttribute("hibernateStart", hStart);
	for (int i = 0; i < 10000; i++){
		user.setUsername(i + "@test.ie");
		userService.create(user);
		user.setEnabled(true);
		userService.updateUser(user);
		userService.delete(user);
	}
	model.addAttribute("hibernateFinish", (System.nanoTime() - hStart));
	return "testdatabases";
}
\end{lstlisting}
\caption{JDBC Vs Hibernate Test}
\label{fig:jdbcvhibernate}
\end{figure}

The results are displayed in Table~\ref{fig:resulttable} below. NS refers to NanoSeconds, and MM:SS refers to the total in minutes and seconds. The test was run three times and an average was created. The test machine was an Amazon EC-2 server, which was a 32-bit micro instance, with 650mb of RAM with one virtual CPU. This is running a modified version of Red Hat Enterprise Linux.

\begin{table}[H]
\caption{Hibernate Vs JDBC Results}
\begin{center}
    \begin{tabular}{ |l | l | l | l | p{2.5cm} |}
    \hline
     Run & Database & No Queries & Total (NS) & Total (MM:SS)\\ \hline
	 1 & Hibernate & 10,000 &198469890740&03:20\\ \hline
	 1 & JDBC & 10,000 &211883464327&03:30\\ \hline
	 2 & Hibernate & 10,000 &188055485374&03:07\\ \hline
	 2 & JDBC & 10,000 &200521721809&03:22\\ \hline
	 3 & Hibernate & 10,000 &188423897510&03:06\\ \hline
	 3 & JDBC & 10,000 &193541735201&03:20\\ \hline
	Avg & Hibernate & 10,000 &191649757878 &03:12\\ \hline
	Avg & JDBC & 10,000 & 201982307112&03:21\\ \hline
    \end{tabular}
\end{center}
\label{fig:resulttable}
\end{table}

\begin{figure}[H]
\begin{lstlisting}
//Excepts from the UserDAO for Hibernate and JDBC actions
//Hibernate
return sessionFactory.getCurrentSession(); //return current session
user.setPassword(passwordEncoder.encode(user.getPassword()));
session().save(user);
session().update(user);	
session().delete(user);

//JDBC
MapSqlParameterSource params = new MapSqlParameterSource();//start create
params.addValue("username", user.getUsername());
params.addValue("name", user.getName());
params.addValue("password", passwordEncoder.encode(user.getPassword()));
params.addValue("gender", user.getGender());
params.addValue("member_type", user.getMember_type());
params.addValue("grade", user.getGrade());
params.addValue("ad_line1", user.getAd_line1());
params.addValue("ad_line2", user.getAd_line2());
params.addValue("ad_city", user.getAd_city());
params.addValue("ad_county", user.getAd_county());
params.addValue("contact_num", user.getContact_num());
params.addValue("em_con_name", user.getEm_con_name());
params.addValue("em_con_num", user.getEm_con_num());
params.addValue("role", "ROLE_MEMBER");
return jdbc.update("insert", params) == 1; //end create
MapSqlParameterSource params = new MapSqlParameterSource(); //start update
params.addValue("enabled", enabled);
params.addValue("username", username);
jdbc.update("update",params); // end update
MapSqlParameterSource params = new MapSqlParameterSource(); //start delete
params.addValue("username", user.getUsername());
jdbc.update("delete",params); //end delete
\end{lstlisting}
\caption{Hibernate-JDBC Code}
\label{fig:jdbchibcode}
\end{figure}

In all tests, Hibernate took less time to perform all actions, though there was not much different between both technologies. However, due to the way Hibernate is configured, it is certainly much easier to develop with. Hibernate required 5 lines of code, shown in Figure~\ref{fig:jdbchibcode}, where JDBC requires 23 lines of code, with 3 different SQL queries, one for INSERT, UPDATE and DELETE. 

By defining and mapping attributes in a database to the class with Hibernate, the DAO classes become much smaller and easier to manage with no impact on performance within the application. In this application, for example, there are more than 70 DAO operations, which would mean 70 or more SQL statements. Use of JDBC would also require the definition of SQL JOINs in relation to both the Timetable and Tournament classes, thereby again increasing complexity within the application. There is more configuration to be undertaken with Hibernate, but the end result is a much easier to manage persistence solution that reduces code complexity.

\subsection{Session and Cookie Management}
One benefit of using a framework such as Spring is that the developer does not need to be overly concerned with defining and checking a session in each controller. In a previous project, the developer was responsible to creating and managing the session, as well as ensuring proper access to the site was maintained. Figure~\ref{fig:manualsession} shows how a session must be manually retrieved from a request object within a servlet.

\begin{figure}[H]
\begin{lstlisting}
@Override
protected void doGet(HttpServletRequest request, HttpServletResponse response)
throws ServletException, IOException {
    PrintWriter out = response.getWriter();
    // Print Generic header then check authenitcation and print HTML
    printHeader(out);
    HttpSession currentSession = request.getSession(false);
    checkAuthentication(out, currentSession);
    // Print search form
}
\end{lstlisting}
\caption{Manual Session Creation}
\label{fig:manualsession}
\end{figure}

In the \textit{checkAuthentication()} method, depicted in Figure~\ref{fig:sessionlevel}, the developer must check the session for any attributes that match a certain value, and act accordingly. In this example, the id of a user. The id then needs to checked against a database to ensure that it is a valid id, and to determine what permissions the account has.


\begin{lstlisting}
private void checkAuthentication(PrintWriter out, HttpSession currentSession) {
//Check Authentication
	if (currentSession.getAttribute("ID") == null) {
		//print registration screen
    } else {
        // Use ID and Username to identify users. At this point we know there is a session with an ID, verify it is a valid session
        int intID = (Integer) currentSession.getAttribute("ID");
        String username = (String) currentSession.getAttribute("username");
        if (customerBean.checkCustomerId(intID) == true && customerBean.checkCustomerUsername(username) == true) {
			Customer currentCustomer = customerBean.getCustomer(intID);
			 // Print what customer should see
            } 
		else if (adminBean.checkAdminId(intID) == true && adminBean.checkAdminUsername(username) == true) {
            Admin currentAdmin = adminBean.getAdmin(intID);
			// Print what admin should see   
            }
        }
	}
\end{lstlisting}
\begin{figure}[H]
\caption{Session Authentication based on user level}
\label{fig:sessionlevel}
\end{figure}

An issue with that is that it is reliant on the developer to identify the correct attribute. This is a user defined attribute that may not be unique to the application. Spring manages this through the \textit{security-context.xml} file, which encapsulates security for the application to one file, and the framework itself ensures that the correct session is created for a user. There is no need to specify any of this within the codebase of the application, which again reduces complexity, and minimises the possibility of security leaks being introduced through developer errors.

\section{Architectural Quality}

(Is there a need for this section? Does the preceding section suffice? Evolution support is covered, security can be measured in resisting, identifying and recovering from attacks \cite{bass2003software}

\subsection{Performance}

\subsection{Security}

\subsection{Evolution Support}


Performance, Security and Evolution Support

\section{Functional and Non Functional Requirement Evaluation}

\section{FYP Process and Learnings}

The FYP has been certainly been significant part of the overall college journey. Looking back, and knowing what I know now, there are a number of avenues that I would have liked to explore.

\subsection{Hibernate in Design}
Hibernate was a very powerful framework throughout the development of this project. It was very much a case of learning by doing. Having accumulated experience with the framework, I would like to apply it more in the design stage on an application. One idea that came to me late in the project was the idea of having a top level class that could build the entire application, as depecited in Figure~\ref{fig:toplevel}.


\begin{lstlisting}
@Entity
public class WebApplication(){

@ElementCollection
@CollectionTable (name = "tournament", joinColumns=@JoinColumn(name="id"))
private Tournaments tournaments;

@ElementCollection
@CollectionTable (name = "timetable", joinColumns=@JoinColumn(name="id"))
private Timetable timetable;

@ElementCollection
@CollectionTable (name = "news", joinColumns=@JoinColumn(name="id"))
private News news;

@ElementCollection
@CollectionTable (name = "events", joinColumns=@JoinColumn(name="id"))
private Events events;

@ElementCollection
@CollectionTable (name = "users", joinColumns=@JoinColumn(name="id"))
private User users;

public WebApplication(){
	create();
}

private void create(){
	tournaments = getTournaments();
	etc.....
}

//getters and setters

\end{lstlisting}
\begin{figure}[H]
\caption{'Top Level' Class}
\label{fig:toplevel}
\end{figure}

(This section needs to be rewritten. I'm tired right now so just doing stream of conciousness) 

It would be possible to build the application so that all classes were linked through a hierarchy to one root class. This may impact performance, but it would allow a number of benefits such as ease of deployment, and the ability to take snapshots of the application at any stage.

I also would have like to use Hibernate with more complicated objects rather than just collection classes, but time constraints did not allow this.

\subsection{Tournament Business Logic}

The tournament section of the site is something that I would like to have done a bit more on, but unfortunately, due to the time constraints within a FYP, a decison had to be made to focus more on either the timetable or the tournament. The timetable provides a service for all members of a club, not just those playing competitively. In the interim, third party solutions such as \textit{http://ti.tournamentsoftware.com} exist. The tournament sorting mechanism was designed so that the logic of sorting members into teams is encapsulated in one method in an interface. Implementing a new version would not require any significant changes to the application. 

\subsection{Web Services and HTML5}
Other aspects of a web application, such as web services and HTML5 were omitted from the final submission due to time constraints, but are definitely something that I would have an interest in looking at later on. Web services were touched on briefly, through the exposure of part of information about the timetable. 

\subsection{Timetable}
While the timetable works well in its current state, I would like to optimize the code to be more readable. I experimented with the use of key-map objects later in the development cycle, and beleive the use of these could reduce the code size in the View by at least 60\%, and help compartmentalize two or three methods into one.